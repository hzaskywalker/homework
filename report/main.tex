\documentclass{article}
\usepackage{overcite}
\usepackage{algorithm}
\usepackage{algorithmic}
\usepackage{subfigure}
\usepackage{listings} 
\usepackage{bm}
\usepackage{multirow}
\usepackage{url}
\usepackage{amsmath,amsfonts,amsthm,amssymb}
\usepackage{fancyhdr}
\usepackage{multirow}
\usepackage{extramarks}
\usepackage{chngpage}
\usepackage{color}
\usepackage{graphicx,float}
\usepackage{indentfirst}
\usepackage{bibentry,natbib}
\theoremstyle{plain} \newtheorem{theorem}{常识}[section]
\theoremstyle{plain} \newtheorem{lizi}{例}[section]
\newcommand{\Class}{Mathematics for Computer Science}

% Homework Specific Information. Change it to your own
\newcommand{\Title}{Homework 3}
\newcommand{\StudentName}{Huang Zhiao}
\newcommand{\StudentClass}{JK 40}
\newcommand{\StudentNumber}{2014011345}

% In case you need to adjust margins:
\topmargin=-0.45in      %
\evensidemargin=0in     %
\oddsidemargin=0in      %
\textwidth=6.5in        %
\textheight=9.0in       %
\headsep=0.25in         %

% Setup the header and footer
%\pagestyle{fancy}                                                       %
\lhead{\StudentName}                                                    %
\chead{\Title}                                                          %
\rhead{\firstxmark}                                                     %
\lfoot{\lastxmark}                                                      %
\cfoot{}                                                                %
\rfoot{Page\ \thepage\ of\ \protect\pageref{LastPage}}                  %
\renewcommand\headrulewidth{0.4pt}                                      %
\renewcommand\footrulewidth{0.4pt}                                      %

%%%%%%%%%%%%%%%%%%%%%%%%%%%%%%%%%%%%%%%%%%%%%%%%%%%%%%%%%%%%%
% Some tools
\newcommand{\enterProblemHeader}[1]{\nobreak\extramarks{#1}{#1 continued on next page\ldots}\nobreak%
                                    \nobreak\extramarks{#1 (continued)}{#1 continued on next page\ldots}\nobreak}%
\newcommand{\exitProblemHeader}[1]{\nobreak\extramarks{#1 (continued)}{#1 continued on next page\ldots}\nobreak%
                                   \nobreak\extramarks{#1}{}\nobreak}%

\newcommand{\homeworkProblemName}{}%
\newcounter{homeworkProblemCounter}%
\newenvironment{homeworkProblem}[1][Problem \arabic{homeworkProblemCounter}]%
  {\stepcounter{homeworkProblemCounter}%
   \renewcommand{\homeworkProblemName}{#1}%
   \section*{\homeworkProblemName}%
   \enterProblemHeader{\homeworkProblemName}}%
  {\exitProblemHeader{\homeworkProblemName}}%

\newcommand{\homeworkSectionName}{}%
\newlength{\homeworkSectionLabelLength}{}%
\newenvironment{homeworkSection}[1]%
  {% We put this space here to make sure we're not connected to the above.

   \renewcommand{\homeworkSectionName}{#1}%
   \settowidth{\homeworkSectionLabelLength}{\homeworkSectionName}%
   \addtolength{\homeworkSectionLabelLength}{0.25in}%
   \changetext{}{-\homeworkSectionLabelLength}{}{}{}%
   \subsection*{\homeworkSectionName}%
   \enterProblemHeader{\homeworkProblemName\ [\homeworkSectionName]}}%
  {\enterProblemHeader{\homeworkProblemName}%

   % We put the blank space above in order to make sure this margin
   % change doesn't happen too soon.
   \changetext{}{+\homeworkSectionLabelLength}{}{}{}}%

\newcommand{\Answer}{\ \\\textbf{Answer:} }
\newcommand{\Proof}{\ \\\textbf{Proof:} }
\newcommand{\Acknowledgements}[1]{\ \\{\bf Acknowledgements:} #1}
\newcommand{\Infer}{\Longrightarrow}
\newcommand{\ud}{\mathrm{d}}
\newcommand{\Reduce}{\Longleftarrow}
\newcommand{\Endproof}{\hfill $\Box$ \\}
\newcommand{\Real}{\mathbb{R}}
\newcommand*\circled[1]{\tikz[baseline=(char.base)]{\node[shape=circle,draw,inner sep=2pt] (char) {#1};}}

%%%%%%%%%%%%%%%%%%%%%%%%%%%%%%%%%%%%%%%%%%%%%%%%%%%%%%%%%%%%%


%%%%%%%%%%%%%%%%%%%%%%%%%%%%%%%%%%%%%%%%%%%%%%%%%%%%%%%%%%%%%
% Make title
\title{\textmd{\bf \Class: \Title}\\\normalsize\vspace{0.1in}}
%\date{} % --- if you de-comment \date{}, the date will not appear below the title
\author{\textbf{\StudentName}\ \ \StudentNumber}
%%%%%%%%%%%%%%%%%%%%%%%%%%%%%%%%%%%%%%%%%%%%%%%%%%%%%%%%%%%%%

\usepackage[colorlinks,linkcolor=red]{hyperref}


\usepackage[]{xeCJK}
%\setmainfont{Courier New} % 设置英文衬线字体
%\setCJKmainfont{} % 设置缺省中文字体


\begin{document}

\title{计算机组成与原理实验报告}
\author{计科40~~黄志翱~~2014011345}
\maketitle

\section{实验要求}
\subsection{一}
在Y86流水线处理器中增加IADDL(iaddl 立即数,目标寄存器)与LEAVE指令。

\subsection{二}
设计实现两个共享内存的Y86流水线处理器(各带私有的L1 Cache)间的数据通信。

\section{实验一}
\subsection{题意及解法分析}
\subsubsection{iaddl}
iaddl的等价代码为。

\subsubsection{leave}
leave的等价代码为,基本操作与pop类似,故照抄。

\section{实验二}
\subsection{题目分析}
显然该问题是由两个完全不同的问题拼凑而成的。

\begin{itemize}
    \item 实现具有write-allocation-write-back的两个l1 cache,并实现两者通讯以达成缓存一致性。
    \item 基于共享内存实现Y86流水线之前的通讯。
\end{itemize}

我们只需要分别解决这两个问题即可。

\subsection{l1 cache}
首先详细地定义我们要实现的问题:在一个unix-like system下,我们需要实现两个进程。
每个进程,需要支持从内存中读取与向内存中写入。

缓存是一组键值对应的表,键指示内存中的位置,值则对应于存储在内存上的内容。

由于需要维护cache,所以向内存读取的方式被限定为:查找被访问的地址是否在缓存中,若在,则读取,否则从内存将相应的内容加载到缓存,读取。

由于要求write-allocation-write-back,所以向内存写入的方式被限定为:查找被访问的地址是否在缓存中,若在,则直接在缓存中写入,否则从内存将相应的内容加载到缓存,写入。另外,当缓存满时,需要将某个缓存中的元素写入内存。

对于多进程的情况下,我们需要加入如下两个约束:

\begin{itemize}
    \item 对于内存中同一位置的写入不能同时进行。
    \item 从缓存中读取时所得到的值必须是正确的。
\end{itemize}

为了优雅,我们还要求不能直接锁死内存,以及尽可能少减少向内存的写入和读取操作。为了实现这个目的,我采取了经典的MESI协议。

\subsubsection{MESI}
在MESI中,我们要求如果有某个元素存在于两个缓存上,则这两个缓存上的内容必须一致。

具体而言,对于一个缓存的一个键值,存在四种状态:

\begin{itemize}
    \item M,该位置有效,键被该进程独占,值与内存中不同。
    \item E,该位置有效,键被该进程独占,值与内存中相同。
    \item S,两个进程都存有此值,且两个缓存,内存中的值都是一致的。
    \item I,此位置的缓存无效。
\end{itemize}

假定两个进程之间能够通讯,现在进程A想做一点事情,那么我们可以很容易地凭借这些状态对内存进行操作:

\begin{itemize}
    \item 若当前位置是M,则A可以直接操作而当做B不存在。
    \item 若当前位置是E,则A可以直接操作而当做B不存在。
    \item 若当前位置是S,如果A是Read操作,可以直接读取而当B不存在,如果A要Write,则A需要通知B放弃该值,将状态改为I。
    \item 若当前位置是I,A需要通知B,若该值在B存在,则B需要将值写入内存,A将其加载进入缓存。如果A是读取,则A,B状态都是S,如果A是写入,则A状态变为M,B状态变为I。
\end{itemize}

\subsubsection{cache通讯}
A, B之间通讯依赖于bus的存在。基本方法是:当A想要对B进行任何信息交流时,A发出信号。A和B在闲暇时刻随时监听对方的信号,一旦B空闲下来了,就发出信号通知A表示应答。此时A即可将所要发出的信息提供给B,A继续等待B接收信息处理。这样通讯直到最后A发出结束信号,继续各干各活。

如果A和B同时发出请求,则让编号大的回答标号较小的防止死锁。

\subsubsection{实现细节}
所有相关代码在misc/cache/中。

server.c提供两个cache共享的内存mem和总线bus。内存通过shmat相关函数获得。内存id由ftok函数指定,id信息被存放在shareinfo.h以供所有文件访问。

cache.c则实现了一个cache。首先根据shareinfo.h申请访问mem和bus。然后利用id得知自己是A还是B(从硬件角度上来讲,两个cache本身一定是有区分的,所以这个id在程序启动时就指定了),开一个共享内存$msg_{id}$,监听$msg_{id}$中是否有读写请求。当收到一个请求后,就利用MESI协议进行操作。监听msg和监听bus交替进行。无论何时收到请求则立刻回答。

\subsection{对psim的修改}
\subsubsection{同时运行两个处理器}
假定两个cache进程已经运行,则cpu往内存进行的读写操作都通过msg发出请求进行即可。从这个角度看,只要cpu通过编号确定与哪个cache连接即可,剩下的操作与单核cpu完全一致。

所以第一步修改就是将isa.c中所有涉及到内存读写的get\_word\_val, get\_byte\_val, set\_word\_val, set\_byte\_val, load\_mem中所有对content的读写替换成通过msg向cache发出读写请求。

值得注意的是,此时如果用两个模拟器同时运行程序的话,load\_mem会将两个代码加载在同一个地方,所以我还修改了相应的函数和读取代码阶段valp的值将代码放置在不同的内存地址。

\subsubsection{原子操作test}
为了使得运行相同代码时代码能够得知自己的身份,我加入了原子操作test:
$$test\ addr,reg$$
效果为读取addr的值,将值赋给reg,同时将addr处的值加1。

改代码格式与mrmovl基本类似,所以在修改pip-full.hcl时只要在所有有mrmovl的地方添加上该操作即可。

但是为了使得这个操作在最低层面是原子的,我们需要让cache也支持这个操作——这是非常容易的,我们可以将对内存地址的test看成read和write的组合,为了保持这两个操作是连续的,只需要在A,B通讯的时候让B再等一下即可。

为了在psim中使得该操作是连续的,理论上只需要将mrmovl读取内存的时候,改为test即可。但是psim.c的实现非常奇怪,操作发生在update函数中,故而我们需要记录下要写入的寄存器的值,在update时进行test并对寄存器赋值。

为了视觉效果,我在test中加入了往屏幕输出test结果的功能。

\subsection{y86 code}

代码流程如下:根据对指定内存的test决定是A还是B,跳到指定的代码段。

A会进行多次循环,每次循环是往连续的内存段中写入1,1的个数从1024到1024-100。写完之后会通过一个内存byte通知B已经写完。等待B回信,继续循环。

B会等待A发出询问,当A询问完毕,会从特定内存位置开始遍历,数1的个数,直到内存中的值为0结束。代码如下:

\begin{verbatim}
.pos 0
irmovl 0, %ecx
test 0x00000f08(%ecx), %eax
addl %eax, %eax
je A
jmp B

A:
irmovl 0, %ecx
rmmovl %ecx, 0x00001001(%ecx)
test 0x00001005(%ecx), %ebx
irmovl 101, %ebx
ALOOP:
    irmovl 0, %ecx
    rmmovl %ebx, 0x00000f00(%ecx)
    AWAIT:
        mrmovl 0x00001005(%ecx), %ebx
        addl %ebx, %ebx
    je AWAIT
    test 0x00001010(%ecx), %edx

    irmovl 1025, %edx
    irmovl 0, %ecx
    irmovl 0x00002000, %eax

    INITA:
        rmmovl %ecx, (%eax)
        irmovl 4, %ebx
        addl %ebx, %eax
        irmovl 1, %ebx
        subl %ebx, %edx
    jne INITA

    mrmovl 0x00000f00(%ecx), %ebx
    rrmovl %ebx, %edx
    irmovl 923, %ebx
    addl %ebx, %edx
    irmovl 0x00002000, %eax
    irmovl 1, %ebx
    irmovl 4, %ecx

    ALOOP3:
        rmmovl %ebx, (%eax)
        addl %ecx, %eax
        subl %ebx, %edx
    jne ALOOP3

    rmmovl %edx, 0x00001005(%edx)
    rmmovl %edx, 0x00001015(%edx)
    irmovl 1, %ecx
    rmmovl %ecx, 0x00001001(%edx)

    irmovl 0, %ecx
    mrmovl 0x00000f00(%ecx), %ebx
    irmovl 1, %ecx
    subl %ecx, %ebx
jne ALOOP
irmovl 0, %ecx
irmovl 2, %ebx
rmmovl %ebx, 0x00001015(%ecx)
halt

B:
BLOOP:
    irmovl 0, %ecx
    BWAIT:
        mrmovl 0x00001001(%ecx), %ebx
        addl %ebx, %ebx
    je BWAIT

    mrmovl 0x00001015(%ecx), %edx 
    irmovl -2, %ecx
    addl %ecx, %edx
    je HALT

    irmovl 0, %ecx
    rmmovl %ecx, 0x00001001(%ecx)

    irmovl 4, %ebx
    irmovl 0, %edx
    irmovl 0x00002000, %eax
    BLOOP2:
        addl %edx, %ecx
        mrmovl (%eax), %edx
        addl %ebx, %eax
        addl %ebx, %edx
        subl %ebx, %edx
    jne BLOOP2

    irmovl 0, %edx
    rmmovl %ecx, 0x00001010(%edx)
    irmovl 1, %ecx
    rmmovl %ecx, 0x00001005(%edx)
jmp BLOOP

HALT:
    halt
\end{verbatim}

\subsection{代码的运行}
进入misc/cache/ make之后,再在pipe/进行make即可运行。

在终端键入python3 run\_two\_sim.py即可运行。

程序会依次从1024输出到925。
\end{document}
